\section{Thesis Statement}

\section*{Idea}
% Why is this study relevant to me? Why is it important to me?
Technology is an integral piece of our lives. Digital literacy is vital in everyday life.
I am a father; I want my children to have a great education.
I have friends who have struggled with math their whole life and I have friends that don't want anything to do with code or math. I think this is detrimental to them because they don't understand that piece of the world around them. 
% What Do I want to accomplish?
I want to help integrate computational thinking into everyday curriculum.
I want to determine for myself if programming can be a viable tool for teaching.
% What questions do I have?
% Are there any other questions that we might investigate?


\subsection*{Research Questions}
\begin{enumerate}
	\item Does programming math provide better transitions for third graders?
	
	I think that it could. As one paper said, programming isn't difficult it just requires clear thinking. 

	\item Is there a better way to teach math (using programming or the more traditional method)? 
	
    I believe that there needs to be deeper understanding. From my conversations with educators it seems that the goal is the same. There have been education reforms with hopes to improve it for all students from primary to post secondary. There is a strong desire to improve but there are some road blocks. 
    
		Those road blocks, I believe, are time limitations; fulfilling a number of requirements imposed by federal, state, and local governments; diverse class sizes, capabilities, and students; lack of funding;
		Many of these roadblocks can be remedied, while others require more thorough investigation, others may not be feasible to change. Lack of funding is difficult but it can be worked around in creative ways. Divers classrooms can be both beneficial and challenging, providing this type of learning activity may be great for some kids but more challenging for others. Time limitations are difficult to work around but again with creativity hopefully we can develop more effective teaching tools to decrease the amount of time required to teach students, decrease the amount of time it takes for students to learn the concepts, and decrease the amount of time teachers need to put toward reviewing and evaluating performance.
	
	\item Can programming be used for more than creating games and telling stories at this age?
	
	I believe it can. Using appropriate levels of scaffolding, programming can be a useful tool to help in specific areas of teaching. As Sherin suggests
	
	\begin{quotation}
			The point here is that programming forces a student to engage with these issues in a way that algebra-physics does not. … it is not too surprising that algebra-physics instruction leaves some undiscovered intuitive cobwebs in this territory, and that the act of programing tends to expose these cobwebs 	p48\cite{sherin}
	\end{quotation}
	\begin{quotation}
		...programming physics draws more strongly on causal intuitions	p50\cite{sherin}
	\end{quotation}
	\begin{quotation}
		...programs might be easier to understand or interpret than equations. 	3\cite{sherin}
	\end{quotation}
	\begin{quotation}
		...the act of programming requires the students to explode each instant of the motion into a series of actions that happen through what I will refer to as ‘pseudo-time’	34\cite{sherin}
	\end{quotation}

	\item How can we better incorporate digital literacy into our education?
    
    Using technology in the classroom helps students become digitally literate, but being able to tell a computer what to do increases said literacy. 
    
\item Determine if visual programming can be used to teach third grade students concepts in mathematics.
\item Does use of visual programming assist students in transitional learning?
\item In what ways does programming, and observing the execution of said program help students understand mathematical concepts?

\end{enumerate}
