
\section*{Introduction}
Transitions move us from where we are to what comes next.
They can be seamless or more difficult depending on the situation.
Elementary school is full of those transitions.
Some students maneuver through these transitions easily while other students find them much more difficult. 
In the latter situation teachers often need to be creative in order to find ways to help facilitate these transitions.

Learning the basics of mathematics plays a key role in developing logical and abstract thinking skills and problem-solving skills.
For some students, mathematical concepts are simple; they just make sense.
For others, these concepts don’t come easily at all.
These preconceptions often originate from a number of reasons but that is a conversation outside the focus in this paper.
The fact is, understanding the complexities of mathematics is difficult for many students to understand.

% Brief Introduction of the education standards
One of the most challenging parts of education is determining what to teach and how to teach it.
Some recent attempts at education reform include the No child left behind Act,  the Every Student Succeeds Act and the Common Core Standards.
Many education reforms, or at least parts of them, introduce positive and promising results while others perform less than optimal.
Regardless of what system is in place, there is a need for teaching tools that help students develop skills that will help them in all aspects of life.
Creativity, problem-solving, critical thinking, and communication are vital for someone to become well rounded, responsible individuals that contribute to their home-life, work, and communities.

% Brief Introduction of third graders
The focus of the proposed study is not to propose a system that outlines education reform nor to argue that there is a need for it.
Its focus is much more specific: third-grade mathematics.
Third-grade is a common time for students to begin learning about multiplication and division, fractions, structures of arrays and area, and describing and analyzing two-dimensional shapes, place value, patterns, and problem-solving, number operations, and measurement and data analysis.
The key principles behind learning these concepts are to make sense of problems and attempt to solve them, reason abstractly and quantitatively, construct viable arguments and critique the reasoning of others, model with mathematics, and to look for patterns.

% Where are they in their cognition development?
By age 8, children start valuing relationships more, think about the future, and become more independent from parents and family.
They can more accurately describe their feelings and understand more about their place in the world.
They are beginning to see and understand things from another person’s point of view.
Their ability to solve problems increases and their attention span gets longer.
In math at school, they should be pretty comfortable with addition and subtraction, and they are becoming more proficient with multiplication.

% Why is the current teaching method insufficient
Technology is increasingly becoming a more integral part of our daily lives.
One might argue that teaching multiplication and division are no longer necessary because technology can perform those operations for us.
Although the latter statement may be true students benefit from skills developed while learning those mathematical operations.
Teaching these topics, or at least the principles behind them, are ever more important because technology often does much of the thinking for us.
Integrating technology and computational thinking will be extremely valuable to students in their current and future everyday life.

% TODO: Source this
One of the major transitions from second-grade to third-grade mathematics is moving from array-based thinking, using addition, subtracted, repeated addition, and repeated subtraction, to a unit-based, more abstract, thinking, using multiplication and division.
%[SOURCE].
This later leads to fractions and ratios.
This transition is often difficult for students and a non-traditional approach may prove beneficial.

% TODO
Another major goal of mathematics at this age is to help the students understand the fundamental pieces of numerical operations and how they are related to one another.
Although quick answer algorithms exist, without a strong understanding of the functions behind them it is difficult for students to truly understand the concept.
If this is the case, students often just plug in the numbers and hope that they turn out.
This may provide correct solutions, but it doesn't lend itself well to solving problems that haven't already been solved.
This is detrimental to subsequent learning because every step builds on the previous.
If this happens, later on, teachers find that they need to fill in the gaps.
This could introduce remedial coursework that may not be needed by other students and may cause disinterest or lost opportunities for deeper learning.

% What are the benefits that could come from this study 
Understanding the aforementioned misconceptions and shortcomings offer us key insights that can help us develop different teaching tools that help students fully understand fundamental concepts.
We propose using programming as a method for teaching multiplication and division at a third-grade level as it could provide an appropriate transition from the group based thinking students have developed from second-grade to the unit based, more abstract thinking they need to develop for future problem-solving skills.
It has the potential to assist students in understanding fundamental concepts and fill learning gaps before they become detrimental to both the individual student’s learning and the learning of their classmates.
It offers an appropriately tailored transition using programming to help them solidify the basics and move to solve more advanced topics.

% What type of study we hope to carry out (quantitative and qualitative)
This study will have a control group and a test group involving two parts, quantitative analysis, and qualitative analysis.
The qualitative portion is built off a pre-test and post-test.
The qualitative analysis will consist of various pieces: recordings of thinking out loud, pre- and post-interviews, and observation of classroom instruction and participation.
