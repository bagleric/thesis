\section{Methods}
% Methodology in a research proposal
Due to the small sample population size and the scope of the study, we don't expect the study to be statistically significant. However, we do expect that it can provide insight for future studies and contribute to integrating computational thinking into the classroom through mathematics.

The study will be carried out during the USU Spring 2021 semester.

\subsection*{Organization}

The study will be divided up into programming activities. Each programming activity will have specific learning objectives that focus on learning a specific education standard. Each activity will also highlight different programming concepts.% debugging, planning, problem solving

These programming activities can be viewed as homework assignments accomplished either in the classroom or at home. 
One or two assignments will be given each week, or every day that a new multiplication or division topic is taught.

\subsection*{Gathering Data}
    The study will be mainly quantitative but it may also involve some qualitative data. 
    It will involve a control group and a test group, preferably taught by the same teacher. %in order to limit the amount of differences in the classroom environment
    The purpose of this study is for academic and practical purposes. It contributes to the well studied topic of using programming as a teaching tool and it provides more specific application to elementary school.

    \subsubsection*{Quantitative Data}
    Qualitative data will be gathered at various points during the study. 
    Before the study students will participate in a pre-test. The purpose of the pre-test is to evaluate where each student stands and if they are meeting the current standards.
    %understanding of specific topics before said topics are taught. This can also help us determine if all of the students are meeting the expected education standards, benchmarks or milestones.

    Before each activity in the study a short questionnaire will be given. 
    %This two or three question survey will help us understand more about how the student feels about the subject area, how they feel about the programming environment, where their stand as far as their understanding, and what their confidence level is.
    % This is mainly helpful juxtapose with post-activity data.

    During the activities we can gather other various forms of data. These may include: the total time it takes for the student to complete the activity, the total number of blocks that they use to complete the activity, the total number of blocks they drag onto the canvas but don't use, the number of times they run the compiler and observe the program running, their final code block solution, and whether or not they successfully completed the activity among other things. % Device (is it touch screen or not), how big is it? 
    %This data, paired with pre- and post-activity data can give us insight as to how they are doing in the coding environment. 
    This data can be used both for observing performance of a single activity and activities over time. 
    If students take much longer than expected it may mean that there are flaws in the programming environment. %: the user interface is too complicated, there isn't enough scaffolding in the program, or that the program is designed poorly. It can also give us insights into their understanding of the math objectives of the activity. For example if in one session a student completes the programming activity faster than the average time, but on a subsequent activity he performs much slower than the average time, it may be that he doesn't understand the math topic and needs more assistance from his parents or teachers. 
    
    After each activity students will complete a second questionnaire for them to reflect on what they learned and to see how they felt about the activity. %This will help us understand how they generally feel after the activity.
    Coupled with pre-activity evaluations we may see trends. % If students for example feel very confident before the activity but leave less confident it may be an indicator that they are more confused, possibly because of the programming environment.  

    After the study is complete a post-test will be given to determine if students met the learning objectives. 

    \subsubsection*{Qualitative Data}
    Qualitative data may include written feedback from the teacher, parents, or the student. This can give us general insight on the study overall. %Where the hypothesis accurate? Does the quantitative data match up with the qualitative data? If so, in what ways? If not, where was the discrepancy?
    
\subsection*{Participants}
    % How will students participate in the study?
    The students will participate in the programming activities. They may carry them out individually, as a class, or in small groups. They will provide feedback about the programming environment, learning activity, and their general experience.

    % How will teachers participate in the study?
    Teachers will introduce the course material and the learning activities. They may also introduce the programming environment. They may provide feedback about the study in general.
    
\subsection*{Example Activity}

\subsection*{Math Topics}
