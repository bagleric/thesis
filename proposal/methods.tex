\section{Methods}
% Methodology in a research proposal
% Research type	
    % Will you do qualitative or quantitative research?
    % Will you collect original data or work with primary or secondary sources?
    % Is your research design descriptive, correlational, or experimental?
% Sources	
    % Exactly what or who will you study (e.g. high school students in New York; Scottish newspaper archives 1976-80)?
    % How will you select subjects or sources (e.g. random sampling, case studies)?
    % When and where will you collect the data?
% Research methods	
    % What tools and procedures will you use (e.g. surveys, interviews, observations, experiments) to collect and analyze data?
    % Why are these the best methods to answer your research questions?
% Practicalities	
    % How much time will you need to collect the data?
    % How will you gain access to participants or sources?
    % Do you foresee any potential obstacles, and how will you address them?


    
\section*{Study}
% When will it happen? What semester? 
We hope to carry out the study during spring semester of 2021.
% What does the study Look like?
% How long will the study be? How many classroom hours will be required? Will it involve home study? 
The study will take place over the course of a few weeks
% How would this study look if it were done in a group face to face setting?
In an in person classroom setting there could be many ways to organize this study. 
One - Teacher lead:
The teacher can lead the students in the programming activity. This may be helpful first starting out to get the general idea of what the programming environment is like and students can ask questions.
Two - Student Lead, Teacher Oversee: 
The teacher can select students to lead the class to solve a certain problem or complete a step. After the problem or step is complete another student can control the computer. Students not controlling can work together to help the controller to write the code.(similar to ladybug programming in \cite{fessakis}).
% How would this study look if it were done individually online?
In an online format with a single user, the program would look similar to the in person version. However, in this case, as student would carry out the exercises by her/himself. 
% Which is preferred or better? Group activities vs individual? (in what ways are each stronger/weaker?)
Both individual and classroom exercises could provide different learning focuses. 

In a classroom setting children may give different facial or verbal queues that indicate the level of their understanding. The teacher could also help students at varying levels of achievement succeed by controlling the activities assigned and which solely activities are observed by each student. It may provide students the opportunity to learn from the experiences of their classmates. A disadvantage of classroom group activities is that each student may not reflect or communicate their actual understanding. This may lead the teacher to think they are either understanding more than they actually are, or that they are understanding less than they display. This may also diminish the motivation of some students as they may not feel comfortable in group settings. 

In an individual setting one advantage is that the activity can be tailored to the individual student. This may be more beneficial for lower performing students as it limits distractions, embarrassing moments in front of their classmates, and allows them to perform on their own timeline. Observing in this situation may be more difficult and the data could be more limited than in the in person setting. 

% What type of compensation will the participants receive?
A small amount of funds will be dedicated to the students, teachers, and researchers, participating in the study. 

% Will teachers need to put much more effort into their class prep?
Recognizing that teachers have limited time to cover materials, one goal of the study is to limit the amount of extra time needed for these lessons by making them easy to incorporate in their current curriculum. This will involve consistent and thorough communication with the teachers as to the learning objectives, general time dedicated to such objectives, and evaluation periods. 
The programming environment should not only have a simple interface that is familiar to students, but also provide appropriate levels of transparency and visibility into students' performance that help educators make appropriate evaluation decisions.

% What technological resources would we have to work with? What assumptions can we make? 
We are assuming that all of the students will have some access to a computer and the internet. This may be on a mobile device (preferably a tablet) or a computer with or without a touch screen.


% \subsection*{Curriculum design}
% \subsection*{Classroom design}
% Which one would help us more effectively?
% Individual option
% Group option
% Entire class option

% What type of data will we get from the study?
\section*{Data}
\begin{enumerate}
    \item Timing data: how long does it take each student to complete an activity.
    \item How many of the tutorials, hints, and other help and scaffolding methods were used by the student. 
    \item Mouse Movements or touch points (possibly, this could tell us what their program usage looks like.)
    \item How many blocks did they use?
    \item How many blocks did they drag on? 
    \item How many blocks did they throw away?
    \item What did their code look like? 
    \item What was the running time of their code?  
    \item Information about there devices can also useful as research shows that touch screen devices provide richer learning experiences.
    \item Basic demographic data (ethnicity, household income, do both parents work in or outside of the home)
    \item Previous education performance of students (this could just be the pre-test).
    \item Pre-evaluation.
    \item Post evaluation.
    \item Verbal feedback during learning activities
    \item Touch or mouse feedback
    \item Questionnaire (about the activity: What did you learn? Was it fun? Why? Do you understand [INSERT MATH CONCEPT] more than before? Explain what [INSERT MATH CONCEPT] means? etc)
\end{enumerate}

How to get the data:
	Pre-test,
	Post-test,
	Thinking out loud / ethnography:
        Record the students using the program: observe their interaction with the program (finger movements, facial expressions, verbal interactions, interactions with classmates)
        
        
% How to analyze data
\subsection*{Data Analysis}
Improvement in pre and post-test
T-test
We don’t expect statistical significance (Why?) the study is too small 
Qualitative assessment (what methods we will use, interview, thinking out loud, etc)

\section*{Participants}
% Who will participate in the study?
Students, Teachers, Researchers, Student's Parents (possibly)
% How will students participate in the study?
The students will participate in the programming activities. They may carry them out individually, as a class, or in small groups. They will provide feedback about the programming environment, learning activity, and their general experience.

% How will teachers participate in the study?
Teachers will introduce the course material and the learning activities. They may also introduce the programming environment. They help observe the students and their progress. They may provide feedback about the study, the programming environment, the learning activities, and their general experience. The teacher works with the researchers to develop the programming environment and learning activities for the students.

% How will I and other researchers participate in the study?
Researchers may provide feedback about the study. With advising from teachers, they prepare the programming environment and learning activities. They also analyze the data and report findings.

% In what ways will teachers be compensated?
% In what ways will students be compensated?
% In what ways will researchers be compensated?
Target audience
Students, educators, computer science community

\section*{Programming Environment}
% What Programming language did we choose to use?
Blockly is developed by Google. It is a platform that allows developers to create apps that use block-based programming. It uses the blocks to create syntactically correct code which can then be used in the application. It can be exported to Javascript, Python, and PHP among others. Developers create the apps “blocks” and then build the app that uses those blocks of code. 

It can provide exportable code, it is open source open-source, extensible, and highly capable. For our purposes, we can use this framework to create blocks to be used in the programming environment. The rest of the programming environment would be a web application.

% What were some programming languages we looked into but chose not to use?
Other programming languages that were considered include the following: Alice, Scratch Blocks, Droplet Editor, snap, Scratch, ScratchJr.

% Why did we choose the Language we did over the others?
All of these have their advantages, however, they do not provide the same benefits as Blockly. Blockly provides the basic structure for building blocks, and the interface for visual programming. It is built to be only a piece of an application. This allows us to customize and tailor the program specifically to our needs. Some of the other languages and programs considered are more complex than are needed for the purposes of this study. None of them, as far as I have researched, can interface with custom programs as Blockly can.

% What are the benefits from using this programming environment?
All in all the main benefits of Blockly include: 
\begin{enumerate}
	\item Provides the library and interface for a visual programming language.
	\item It is meant to be a piece of a larger, custom application.
	\item It allows for a simple, clean user interface.
	\item It, in itself is customizable.
\end{enumerate}

% What are the risks that come from using this programming environment?
Although it has many benefits, there also comes a cost. As mentioned it is meant to be a piece of the larger picture. That means there is a possibility that building the application around it could be more intensive than anticipated. If teachers have also used other languages before there may be quite a learning curve (which we hope to flatten through a carefully designed user interfaces). 

% How much time will be required on my part to develop this programming environment?
I anticipate that a large portion of my research and thesis coursework will be dedicated to developing the programming environment. Appropriate time estimations would be more accurate after more design work has been completed. 


Design of language
A scaffolded visual programming language.

Guided learning
Transitions through “levels” with Increasing complexity
Follows pedagogical practices of learning objectives,  with activities that can be adjusted to support those learning goals.

Programming blocks can be limited (you only get 2 or 10 or you can have as many as you need) This could be useful in providing hints to students.

Adaptive programming blocks that can be customized (show this field, don’t show this field) to the learning objectives.

\section*{Math Topics}
% What Math topics could be covered in the study?
Mathematic topics that could be addressed in this study may include the following gathered from the Utah Mathematics Core Guidelines
\subsection*{Reason with shapes and their attributes}
\subsubsection*{Understanding shapes}
Understand that shapes in different categories (for example, rhombuses, rectangles, and others) may share attributes (for example, having four sides), and that the shared attributes can define a larger category (for example, quadrilaterals). Recognize rhombuses, rectangles, and squares as examples of quadrilaterals, and draw examples of quadrilaterals that do not belong to any of these subcategories. 
% \begin{enumerate}
% 	\item Understand that squares, rectangles, rhombuses, parallelograms, and trapezoids are examples of quadrilaterals
% 	\item Compare and contrast squares, rectangles, rhombuses, parallelograms, and trapezoids
% 	\item Identify and draw quadrilaterals that cannot be classified as squares, rectangles, rhombuses, parallelograms, or trapezoids
% 	\item Recognize and understand that the larger category of quadrilaterals includes other subcategories such as squares, rectangles, rhombuses, parallelograms, and trapezoids; Identify examples and non-examples of squares, rectangles, rhombuses, parallelograms, and trapezoids;
% 	\item Recognize that there are quadrilaterals that are not in any of the subcategories
% \end{enumerate}
% How might the programming activity look?
One example of a more advanced programming activity could be to write a program that can draw or generate different shapes.
% How will we determine if math is being learned in each activity? What evidences will lead us to believe it is happening?
If the students are able to write a program that can create one of the shapes above it is a good indication that they understand the shapes. 

\section*{Verification and Results}
% What data will there be to analyze?
% How will that data be analyzed?
% Why is this data useful? What do we hope to get from it?


\section*{Other Risks and Discussion topics}
% What would we do if we can't get a teacher on board with us?
\subsection*{Inability to find a teacher}
If we aren't able to find a teacher that would be willing to hold the study as a part of their class, there may be several alternative approaches.

\begin{enumerate}
	\item Form an after school or online out of school group that students may elect to attend.
	\item Make the programming environment available to the public. Many students and/or their parents would be interested in additional activities that can fortify their understanding. This may require some sort of advertizing or to still work through the schools, but it wouldn't require teacher involvement.
	\item Attempt to look outside of the normal school locations. This may require additional approvals, and more extensive traveling but it may still be doable.
	\item Select a different grade (preferably one close to the original) if there are teachers that would be willing to welcome the study into their classroom
\end{enumerate}
Each of these alternatives would require some extra work and planning. But, should the preferred plan fall through there are a number of other options. 

% How might we need to adjust the study if we cannot meet face to face because of COVID-19
\subsection*{COVID-19 continues through the spring}
COVID-19 does not necessarily pose a risk right now. It may adjust the location of the study from the classroom to student's homes, but the bulk of the study would remain the same. The shortcomings of this approach is that there may not be as much qualitative data to analyze. The lack of qualitative data could limit the findings, weaken conclusions, or leave gaps and unanswered research questions.


