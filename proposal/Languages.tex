Languages

\section{Alice}
\url{https://www.alice.org/}

Alice has been around since 1995. It started out as a rapid prototyping application for live publishing VR experiences. Its main goal was to provide the experience to non-programmers. Three-dimensional development is a key part of Alice’s heritage.  In 1999 the first major release of Alice introduced “drag and drop” coding, born from the idea that syntax is difficult for many audiences to understand. It entered the world as an educational platform around 2004, where they incorporated programming pedagogy into the software. One of their goals down the road a few years was to engage more girls in early computer science. This lead to “Looking Glass” which focused mainly on creating and telling stories. This is part of the current program. Alice 3.0, that focuses on telling stories and animating them, creating games, and learning to program through block-based coding. 

Pros: This is a block-based language and it is one known by Dr. Edwards.
Cons: It seems to be more focused on storytelling and creating games in order to teach programming. Our main goal is not to teach programming but to help students understand mathematics using programming and simulation.



\section{Blockly }
\url{https://developers.google.com/blockly/}

Blockly is developed by Google. It is a platform that allows developers to create apps that use block-based programming. It uses the blocks to create syntactically correct code which can then be used in the application. It can be exported to Javascript, Python, and PHP among others. Developers create the apps “blocks” and then build the app that uses those blocks of code. 

Pros: On their website, they mention that the strengths are: exportable code, open-source, extensible, highly capable, international. For our purposes, this may be useful to build an app that uses the blocks. That would give us the advantage of building our own app and limiting the complexities of other languages from the user. 
Cons: The downside is that we would need to build the actual app to carry out the study. That may be more intensive that we want.



\section{Scratch Blocks}
\url{https://scratch.mit.edu/developers}

Built on collaboration by Google and MIT, and built on Google’s existing “Blockly technology” 

Pros: one of its main design principles is “Many Paths, Many Styles” which translates to the desire to provide more than one way to learn things. Since we are moving away from the traditional methods of teaching Math, specifically division, we share this common goal.

Cons: It is currently in “developer preview” stages so it may not be stable.



\section{Droplet Editor}
\url{https://github.com/PencilCode/droplet}
http://droplet-editor.github.io/

This project focuses on the transition from “block-based” to “text-based” languages. 

Pros: An awesome concept

Cons: Not really what we are looking for



\section{Snap}
\url{https://snap.berkeley.edu/about}

This application is a visual blocks-based programming language. It allows users to create blocks, and then use them in other block programs. It was inspired by Scratch but differs in that lists are first-class data types. This allows for constructing many other complex data structures.

Pros: This allows for complex data structures

Cons: It may be too complex for our audience. This could complicate the learning process for third and fourth graders.



\section{Scratch}
\url{https://scratch.mit.edu/about}

Designed especially for ages 8-16, it utilizes visual programming with the end goal to teach how to code and code to learn.  This application is a visual blocks-based programming language. It allows users to create blocks, and then use them in other block programs. It was inspired by Scratch but differs in that lists are first-class data types. This allows for constructing many other complex data structures.

Pros: This allows for complex data structures

Cons: It may be too complex for our audience. This could complicate the learning process for third and fourth graders.



\section{Logo}
\url{https://en.wikipedia.org/wiki/Logo_(programming_language)}

Logo was designed by Wally Feurzeig, Seymour Papert, and Cynthia Solomon as an educational programming language. The main propose of the language was to teach concepts of programming related to Lisp, where students could understand, predict, and reason about the turtle’s motion. The turtle would carry out positional adjustment commands relative to itself. It was first used in 1968. 



\section{ToonTalk}
\url{https://en.wikipedia.org/wiki/ToonTalk}

Summary

Pros: 

Cons: 

Squeak Etoys
url

Summary

Pros: 

Cons: 

Microworlds
url

Summary

Pros: 

Cons: 

Stagecast creator Jr, 
url

Summary

Pros: 

Cons: 

