
\section{Literature Review}
% for help or details see: https://www.scribbr.com/dissertation/literature-review/

% TODO comment out section heading before finalizing
% \subsection{Introduction} 
% What are the themes

% What are the debates/conflicts

% What are the gaps -- room for future study

% TODO comment out section heading before finalizing
% \subsection{Body} 
% TODO choose an outline structure: Chronological, Thematic, Methodical, or Theoretical 

% TODO comment out section heading before finalizing
% \subsection{Conclusion} 




%When did programming as a teaching method start?
%Who pioneered it and what was their main objective?
%How has it evolved through the years?
% \cite{sherin}
%What research in this area has taken place in the last 10 years?
%What is the main focus of this research?
%What are the findings of this research?
%Are there still unanswered questions? What are the areas of Further study mentioned in the papers?

Using programming as a teaching method for topics outside of computer science is not novel.
There have been many studies related to pedagogy and programming across all educational levels.
Within each level, a wide range of topics are covered.
Computer Science principles are used, however, to facilitate the learning process.
These principles include computational thinking, problem-solving, planning, and creativity.
The aforementioned principles have been used to teach language, writing, mathematics, physics, and other science topics.

% \subsection*{Learning to Code}
The studies reviewed can be placed into three categories: ones that teach computer science principles, ones that teach programming languages, and those that use computer science principles and programming languages to teach subjects outside of computer science.

% TODO add source
The first area addressed in a number of studies have research objectives that focus on determining appropriate ways to teach a specific computer science principle.
Other studies focus on determining appropriate ways to teach programming syntax such as Phanon [EDWARDS].
Based on the target audience there is a variety of teaching software and a variety of different languages that aim to teach different principles.
For example, block-based visual programming targets younger or more novel students, mixed block-text based programming languages such as Droplet can be used for older or familiar students, and text based languages are more easily understood by secondary and post-secondary students.
Each of these have varying levels of scaffolding.
Block programs heavily utilize scaffolding by providing hints and guiding programmers develop code using close to natural language while text based languages can have little to no scaffolding.

Some programs aim to provide dynamic scaffolding that offers more intensive supports initially but gradually removes some or all scaffolding support for more advanced users.
    
\subsection*{Learning through Code}
The second area addressed in a number of studies is using software programs or programming languages as tools to teach educational subjects outside of computer science.
Different than virtual manipulatives or other computer games, programming has proved to be beneficial in helping students develop problem-solving, planning, meta-cognition and other skills \cite{papadakis, fuchs}.

% \subsection{Coding blocks}\cite{}
% \subsection{Programming Physics}\cite{sherin}

% \subsection{Science Topics} \cite{guzdial}
% \subsection{Story telling and Language} \cite{}
% \subsection{Mathematics} \cite{}
% \subsection{Virtual Manipulatives}\cite{lye}

% It is widely known that programming can be used as a successful teaching tool.
% There have been several studies focused on teaching mathematics, many of them use programming to do such.
% The age group of said studies ranges from preschool to post-secondary.
