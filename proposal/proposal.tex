\documentclass[12pt]{extarticle}
\usepackage[utf8]{inputenc}
\usepackage{cite}

\title{Programming as a teaching tool in third-grade mathematics }
\author{Eric Bagley}
\date{June 2020}

\begin{document}

\maketitle

\section*{Introduction}
Transitions move us from where we are to what comes next. They can be seamless or more difficult depending on the situation. Elementary school is full of those transitions. Some students maneuver through these transitions easily. Other students find them much more difficult and teachers often need to be creative in order to find ways to help facilitate these transitions.

Learning the basics of mathematics plays a key role in developing logical and abstract thinking skills and problem-solving skills. For some students, mathematical concepts are simple; they just make sense. For others, these concepts don’t come easily at all. These preconceptions often originate from a number of reasons but that is a conversation outside the focus in this paper. The fact is, understanding the complexities of mathematics is difficult for many students to understand.

% Brief Introduction of the education standards
One of the most challenging parts of education is determining what to teach and how to teach it. Some recent attempts at education reform include the No child left behind Act,  the Every Student Succeeds Act and the Common Core Standards. Many education reforms, or at least parts of them, introduce positive and promising results while others perform less than optimal. Regardless of what system is in place, there is a need for teaching tools that help students develop skills that will help them in all aspects of life. Creativity, problem-solving, critical thinking, and communication are vital for someone to become well rounded, responsible individuals that contribute to their home-life, work, and communities.

% Brief Introduction of third graders
The focus of the proposed study is not to propose a system that outlines education reform nor to argue that there is a need for it. Its focus is much more specific: third-grade mathematics. Third-grade is a common time for students to begin learning about multiplication and division, fractions, structures of arrays and area, and describing and analyzing two-dimensional shapes, place value, patterns, and problem-solving, number operations, and measurement and data analysis. The key principles behind learning these concepts are to make sense of problems and attempt to solve them, reason abstractly and quantitatively, construct viable arguments and critique the reasoning of others, model with mathematics, and to look for patterns.

% Where are they in their cognition development?
By age 8, children start valuing relationships more, think about the future, and become more independent from parents and family. They can more accurately describe their feelings and understand more about their place in the world. They are beginning to see and understand things from another person’s point of view. Their ability to solve problems increases and their attention span gets longer. In math at school, they should be pretty comfortable with addition and subtraction, and they are becoming more proficient with multiplication.

% Why is the current teaching method insufficient
Technology is increasingly becoming a more integral part of our daily lives. One might argue that teaching multiplication and division are no longer necessary because technology can perform those operations for us. Although the latter statement may be true students benefit from skills developed while learning those mathematical operations. Teaching these topics, or at least the principles behind them, are ever more important because technology often does much of the thinking for us. Integrating technology and computational thinking will be extremely valuable to students in their current and future everyday life. 

% TODO: Source this
One of the major transitions from second-grade to third-grade mathematics is moving from array-based thinking, using addition, subtracted, repeated addition, and repeated subtraction, to a unit-based, more abstract, thinking, using multiplication and division. %[SOURCE].
This later leads to fractions and ratios. This transition is often difficult for students and a non-traditional approach may prove beneficial.

% TODO
Another major goal of mathematics at this age is to help the students understand the fundamental pieces of numerical operations and how they are related to one another. Although quick answer algorithms exist, without a strong understanding of the functions behind them it is difficult for students to truly understand the concept. If this is the case, students often just plug in the numbers and hope that they turn out. This may provide correct solutions, but it doesn't lend itself well to solving problems that haven't already been solved. This is detrimental to subsequent learning because every step builds on the previous. If this happens, later on, teachers find that they need to fill in the gaps. This could introduce remedial coursework that may not be needed by other students and may cause disinterest or lost opportunities for deeper learning.

% What are the benefits that could come from this study 
Understanding the aforementioned misconceptions and shortcomings offer us key insights that can help us develop different teaching tools that help students fully understand fundamental concepts. We propose using programming as a method for teaching multiplication and division at a third-grade level as it could provide an appropriate transition from the group based thinking students have developed from second-grade to the unit based, more abstract thinking they need to develop for future problem-solving skills. It has the potential to assist students in understanding fundamental concepts and fill learning gaps before they become detrimental to both the individual student’s learning and the learning of their classmates. It offers an appropriately tailored transition using programming to help them solidify the basics and move to solve more advanced topics.

% What type of study we hope to carry out (quantitative and qualitative)
This study will have a control group and a test group involving two parts, quantitative analysis, and qualitative analysis. The qualitative portion is built off a pre-test and post-test. The qualitative analysis will consist of various pieces: recordings of thinking out loud, pre- and post-interviews, and observation of classroom instruction and participation.  

\section*{Related Work}
Using programming as a teaching method for topics outside of computer science is not novel. There have been many studies related to pedagogy and programming across all educational levels. Within each level, a wide range of topics are covered. Computer Science principles are used, however, to facilitate the learning process. These principles include computational thinking, problem-solving, planning, and creativity. The aforementioned principles have been used to teach language, writing, mathematics, physics, and other science topics. 

% \subsection*{Learning to Code}
The studies reviewed can be placed into three categories: ones that teach computer science principles, ones that teach programming languages, and those that use computer science principles and programming languages to teach subjects outside of computer science. 

% TODO add source
The first area addressed in a number of studies have research objectives that focus on determining appropriate ways to teach a specific computer science principle.
Other studies focus on determining appropriate ways to teach programming syntax such as Phanon [EDWARDS].
Based on the target audience there is a variety of teaching software and a variety of different languages that aim to teach different principles. For example, block-based visual programming targets younger or more novel students, mixed block-text based programming languages such as Droplet can be used for older or familiar students, and text based languages are more easily understood by secondary and post-secondary students. Each of these have varying levels of scaffolding. Block programs heavily utilize scaffolding by providing hints and guiding programmers develop code using close to natural language while text based languages can have little to no scaffolding.

Some programs aim to provide dynamic scaffolding that offers more intensive supports initially but gradually removes some or all scaffolding support for more advanced users. 

    
\subsection*{Learning through Code}
The second area addressed in a number of studies is using software programs or programming languages as tools to teach educational subjects outside of computer science. Different than virtual manipulatives or other computer games, programming has proved to be beneficial in helping students develop problem-solving, planning, meta-cognition and other skills \cite{papadakis, fuchs}.   



\subsection{Coding blocks}\cite{}
\subsection{Programming Physics}\cite{sherin}

\subsection{Science Topics} \cite{guzdial}
\subsection{Story telling and Language} \cite{}
\subsection{Mathematics} \cite{}
\subsection{Virtual Manipulatives}\cite{lye}



It is widely known that programming can be used as a successful teaching tool (. There have been several studies focused on teaching mathematics, many of them use programming to do such. The age group of said studies ranges from preschool to post-secondary. 

The basis of Sherin’s study is to answer the question in what ways are programming-physics and algebraic physics different, and can programming-physics be a viable tool for teaching complex concepts in physics. He compares it to algebraic physics by outlining the key characteristics and interpretive devices (problem-solving tactics and analytic methods) of both. The outcome of his study is that programming physics highlights different interpretive devices and it can be used as a viable physics teaching method.

Guzdial explains the use of programming to teach science topics. He does this using a tool he created called Emile. Emile incorporates scaffolding to assist students in learning the skill of programming in order to not distract from the greater goal of teaching scientific topics. He says “Students use Emile to program models… and simulations … in kinematics. Emile supports students without previous physics or programming background to successfully create models of kinematics and execute these models as simulations and learn about physics in the process.” (p2). He outlines what scaffolding is, how fading works, and what his pedagogy and teaching patterns looked like in his study. Conclusions drawn from the study were that Emile was a successful implementation of software realized scaffolding. It both facilitated in teaching programming concepts while also teaching physics modeling through simulation.

[Papadakis] This study states that research has found that “teaching programming to young children has a crucial influence on the development of their cognitive functions.” They did this using ScratchJr, a block-based programming software that was developed primarily for young children (aged 5-7). They found that preschoolers and young primary schoolers can code and that programming can be used as a teaching tool for other subject areas.

[Shumway, et-al] This paper discusses using virtual manipulative representations as well as physical manipulatives and textbook resources in teaching mathematics to elementary-aged children. The findings suggest that both can be viable teaching tools. They found, however, that the two offered different strengths, indicating that using both provides a more comprehensive learning experience.

The lady bug study 

\section*{Proposal}
% Describe objective
% Research Objectives
\subsection*{Hypothesis}
\subsection*{Research Objectives}
\subsection*{Research Questions}

Determine if visual programming can be used to teach third grade students concepts in mathematics.
Does use of visual programming assist students in transitional learning?
In what ways does programming, and observing the execution of said program help students understand mathematical concepts?
Target audience
Students, educators, computer science community
Design of language
A scaffolded visual programming language. 
Guided learning
Transitions through “levels” with Increasing complexity
Follows pedagogical practices of learning objectives,  with activities that can be adjusted to support those learning goals. 
Programming blocks can be limited (you only get 2 or 10 or you can have as many as you need) This could be useful in providing hints to students. 
Adaptive programming blocks that can be customized (show this field, don’t show this field) to the learning objectives.

\subsection*{Curriculum design}
\subsection*{Classroom design}
Which one would help us more effectively?
Individual option
Group option
Entire class option
How to get the data
	Pre-test
	Post-test
	Thinking out loud / ethnography
		Record the students using the program: observe their interaction with the program (finger movements, facial expressions, verbal interactions, interactions with classmates)
% How to analyze data
\subsection*{Data Analysis}
Improvement in pre and post-test
T-test
We don’t expect statistical significance (Why?) the study is too small 
Qualitative assessment (what methods we will use, interview, thinking out loud, etc)
\section*{Conclusion}
% What is a big impact? 
\subsection*{Implications}
Students will learn math
Computational thinking in real-world situations
What would happen if the study doesn’t work

Shortcomings that might lead to a failed state
What happens if we don’t find a teacher
Other risks needed to mitigate
COVID 19 updates (how would this look online?)
\cite{sherin}
\bibliographystyle{plain}
\bibliography{../thesis}

\end{document}
